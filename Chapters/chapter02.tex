\chapter{Magnetostatics}
The study of \textit{magnetostatics} concerns the study of \textit{time independent} magnetic fields. However, despite  what the name suggests, that does not necessarily mean things aren't moving. With that in mind, we introduce
\section{The Magnetic Force}\ref{sec:magf}\hypertarget{magf}
Contrary to \hyperlink{coulomb}{the electric force}, from which we defined the electric field, the magnetic force by itself already depends on the existence of a magnetic field (more on this later) and is governed by the equation:
\begin{eq}{Lorentz Force Law}{2.1}
	For a charge $Q$ moving with velocity $\mathbf{v} $ along a magnetic field $\mathbf{B} $:
	\[
	\mathbf{F} = Q(\mathbf{v}\times \mathbf{B}  )
	\] 
	where '$\times$' is the vector cross product. And if an electric field $\mathbf{E} $ is also present:
	\[
	\mathbf{F} = Q(\mathbf{E} + \mathbf{v} \times \mathbf{B} )
	\] 
\end{eq}
Once again we have experimentally determined that the magnetic force obeys the \textit{principle of superposition}.\\
Note that, in the absence of an electric field, $\mathbf{F} $ will is a vector that's perpendicular to $\mathbf{v} $ and $\mathbf{B} $ scaled by $Q$. Thus, its direction is determined by the right-hand rule.

\section{The Magnetic Field}\label{sec:magfield}\hypertarget{magfield}
One thing that makes magnetic fields slightly more complicated than electric fields is the absence of \textit{magnetic monopoles}, that is, magnets always come in dipoles.\\
{\color{Red}*** TODO: add figure ***}\\
This means we can't define the magnetic field in terms of the magnetic force as we did in Section \ref{ef}. 
Instead, we've observed that magnetic fields\footnotemark \footnotetext{Or what we may define as \textit{field behavior}, to be pedantic.} are generated by \textit{moving} charges. For now, we keep these charges moving at a constant speed to ensure our magnetic field is non time variant. Recall that charges moving across in a path is what we call \textit{current}. With this in mind, we define the magnetic field by:
\begin{eq}{Biot-Savart's Law}{2.2}\label{bslaw}\hypertarget{bslaw}
	For a steady current $I$ (in Amperes) moving across a path $\mathcal{C}$:
	 \[
		 \mathrm{d} \mathbf{B} = \frac{\mu_0 I}{4\pi} \int_{\mathcal{C}} \frac{\mathrm{d} \mathbf{l} \times \hat{\mathbf{r} }}{r^{2}}
	\] 
	where $\mathrm{d} \mathbf{l} $ is a segment of path $\mathcal{C}$ and $ \hat{\mathbf{r} }$ is the vector from the current source to a point.
\end{eq}
